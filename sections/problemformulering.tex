\chapter{Problemformulering}\label{ch:problemformulering}
I problemanalysen er der udarbejdet en afgrænsning af projektets fokuspunkt. Dette fokuspunkt kommer til udtryk i problemformuleringen. I kapitel 3 blev det centrale problem belyst. Her var problemet, at stormfloder i Danmark forårsager skader. Disse skader har omkostninger for forsikringsselskaberne, som skal dække skaderne. Stormfloder forekommer også hyppigere i takt med, at den globale vandstand bliver forhøjet, grundet smeltningen af glesjere. For at formindske skaderne og aflaste forsikringsfirmaerne, fokuserer projektet på, at forbygge og begrænse skaderne som stormfloderne forårsager. 
\par
I kapitel 4 blev der undersøgt, hvilke muligheder der var for at gøre dette og hvilke løsninger der allerede findes. (Mere omkring eksisterende løsninger). Her ses et potentiale for at automatisere bygning af sandsækkediger ved brug af droneteknologi.
\\\\
På baggrund af problemanalyse og afgrænsningerne udmunder projektets problemformulering:

\textit{Hvordan udvikles et \textbf{værktøj}, som \textbf{simulerer} \textbf{dronesværmes} opbygning af sandsækkediger?}
\\\\
Nøgleord i problemformuleringen beskrives således:

\textbf{Værktøj:}
Simulation skal bruges som en supplering til en fysisk løsning af beskrevne droner.

\textbf{Simulation:}
En computersimulation med algoritmer og grafik, som giver et miljø, hvor bevægelse af agenter kan observeres.

\textbf{Dronesværm:}
For at processen går hurtigere er der brug for en sværm af droner, da en enkelt drone tager for lang tid om arbejdet.

En simulation skal udvikles på grund af manglende ressourcer. Her bliver projektets mål at udvikle bevægelsesmønsteret for de fysiske droner.(mere beskrivende tekst

%\textit{Hvordan udvikles et værktøj, som kan simulere droners opførelse således, at dronerne begrænser mængden af de skader, der forårsages af stormfloder som rammer Danmark ved hjælp af sandsækkediger?}


%I kraft af at klimaforandringerne vil forværre vejret i fremtiden, og oversvømmelser derfor vil stige i frekvens, er det et problem at beskytte huse mod den høje vandstand.

%Forbedring af eksisterende metoder til forsvar mod oversvømmelse



