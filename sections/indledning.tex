\chapter{Indledning}\label{ch:introduction}

% Vi sporer os ind på sværme-fænomenet med nogle konkrete eksempler
I naturen optræder flere dyrearter i sværme: Fiskestimer undviger hajer med deres mange øjne og desorienterende bevægelser \cite{fishschooling}. Enorme flokke af stære blokerer tilsyneladende solen \cite{sortsol}. Hærmyrer bygger broer af hinandens kroppe for at krydse vandløb \cite{armyants}. 
Sværmeadfærd tjener et formål for dyrenes overlevelse: Alt fra undgåelse af rovdyr til nye bevægelsesmuligheder, som eksemplificeret \cite{biofoundations}.\\
% Sværme er relevante at vide noget om
Den adaptive adfærd, som sværme i naturen udviser, har fået forskere til at overveje at implementere viden om den til menneskelige formål.
Indsigt i sværmdynamik tænkes bl.a. at være relevant i beskyttelsen af økosystemet: Flere sværmende insektarter med vigtige roller i bestøvningen af blomster m.m. er på vej til at uddø  \cite{ripinsects}. Forståelsen af problemet og dets mulige løsninger kræver en forståelse af sværme \cite{swarmdecline}. \\ Andre ser muligheder i at oversætte fra natur til teknologi: Bl.a. at lade sig informere af dyrs sværmadfærd i designet af selvorganiserende pixels til højdefinitionskameraer og flyvende drone-formationer til militæret \cite{camera}\cite{pentagon}.
Der er altså flere mulige gevinster for mennesker ved at forstå naturens sværmadfærd.\\\\
% Sværme er svære at forstå i detaljer
Indsigt i de lokale interaktioner mellem dyrene er en nødvendighed for forståelsen af sværmen som helhed. Sværmens adfærd styres nemlig af disse og ikke af nogen central dirigent \cite{selforganimals}.  
Imidlertid kan det være svært at beskrive sværme på dette niveau.\\ Dette kan for det første skyldes praktiske problemer med at indsamle det nødvendige data i felten såvel som i laboratoriet: Sværmdynamik kan eksempelvis være svært at observere i detaljer pga. \cite{swarmsaredifficult}. 
Et andet problem lægger sig til den \textit{ikke-lineære} dynamik, som typisk danner grundlag for sværmadfærd: Her kan systemet som helhed ikke beskrives som summen af de lokale interaktioner mellem dyrene - I daglig tale, er summen noget andet end dens andele\cite{nonlinear}\cite{swarmsarenonlinear}. 
Dette gør det uhensigtsmæssigt at beskrive sværmen \textit{lineært}: Alene ud fra opremsningen af en række faktorer. Kortlægningen af sværmdynamik kan altså være vanskelig i sagens natur.
% Computersimulationer som muligt alternativ til felt- og laboratoriestudier "mv."
Disse udfordringer gør det relevant at undersøge alternative metoder. \textit{Computersimulering} af sværme er en relativt ny tilgang med flere fordele ift. felt- og laboratoriestudier: Udover at adgangen til data om sværmen kan være langt mere ligefrem på en computer, er forbruget af tid og penge pr. forsøgsrunde relativt lavt, så snart systemet er sat op \cite{simulationintro1}\cite{simulationintro2}. Muligheden for et større antal forsøgsrunder gør det alt andet lige nemmere at kortlægge ikke-lineær dynamik. Derudover kan forskellige etiske fordringer ift. forsøg med levende dyr tilsidesættes, og derfor opstår der helt nye muligheder \cite{animalethics}. Tilgangen synes altså at være værd at undersøge nærmere.\\\\
% Den initierende undren
Vores initierende undren for dette projekt er hermed: Hvordan kan man ved hjælp af computersimulering finde indblik i, hvad der danner grundlag for sværmeadfærd i naturen? Dette tekniske problem er relevant at belyse pga. de ovennævnte mulige gevinster ved en bedre forståelse af sværmdynamik. Disse involverede både nye muligheder for at beskytte økosystemet og for ny teknologi. \\
% Meta-kommunikation
Problemanalysen er struktureret således: Først vil sværmeadfærd i naturen blive undersøgt for at indkredse et relevant fænomen, der vil være fokus for projektet.
Efter fundet af et bestemt sværmfænomen i naturen, bliver der foretaget et begrundet valg af tilgang til at implementere simulationen.  
På baggrund af problemanalysen kan der herefter fremsættes en problemformulering, som angiver rammer for projektets løsningsdel.




%Sværm, oversættelse fra natur til computersimulation
% Initierende undren:
    % Hvordan kan man ved hjælp af maskinlæring finde indblik i, hvilke parametre og faktorer der får en bestemt sværm til at opstå? Et delproblem hertil er, at finde hvilke parametre og faktorer i naturen, som har indflydelse på dette.

%Oversættelse fra natur til maskine \\
%Hvad kan det bruges til? (Inspiration fra naturen) \\
%Valg af simulering, som undersøgelsesmetode.\\