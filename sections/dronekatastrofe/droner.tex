% Hvorfor har man valgt at erstatte mennesker i disse situationer?

\section{Droner i forbindelse med naturkatastrofer}\label{sc:dronerkatastrofe}
% Droner bruges i dag til flere formål, når det kommer til naturkatastrofer, blandt andet bruges de til at få overblik over situationer hvor det ville være farligt at sætte mennesker ind, som f.eks. at kortlægge en kollapset bygning og se om der stadig er mennesker inde i den \par \cite{Apvrille2015}\cite{Tanzi2014}, til at få overblik over et område hvor der kan være spildt kemikalier eller noget helt andet.

Udover droners brug til militære og kommercielle formål, bruges de også i nødsituationer. Blandt andet kan de bruges til kortlægning af områder med spildte kemikalier eller til at hjælpe med overvågning af diger og områder der er oversvømmede. 

\subsection{Spredning af farlige materialer}
Et eksempel på, hvordan droner bruges i forbindelse med uheld, der involverer spredning af farlige og radioaktive materialer. I disse situationer spiller droner en vigtig rolle i form af at få overblik over i hvilken retning uheldet spredes. Dermed kan man få evakueret civile i risikozonen, før det er for sent. Dette eksempel viser, hvorfor droner er smarte i sådanne situationer, da man ikke behøver at sende mennesker ind til giftige materialer, og dermed minimerer risikoen for at miste liv.
\par
Udover at få overblik over situationen, kan droner også spille en stor rolle i at søge efter mennesker der er fanget i de ramte områder. Således kan man igen formindske risikoen for at bringe eventuelle redningshold i fare, da de med hjælp fra dronerne kan finde direkte ud af hvor menneskerne kunne være. På denne måde behøver redningsholdet kun at hente personerne og ikke at søge efter dem. 

\subsection{Oversvømmelser}
Et andet eksempel på, hvad droner bruges til i dag er oversvømmelser. Her bruges dronerne både til før og efter hændelsen. Måden hvorpå droner bruges før en oversvømmelse, er at holde øje med dæmninger, og i tilfælde af usædvanlige forhold, kan de ansvarlige autoriteter få informationen og nå at reagere, inden situationen udvikler sig. Dog er droner lidt udfordrede på denne front, da de er begrænsede af blandt andet distancen fra, hvor de bliver styret, men også af batterilevetid. Da dronerne har disse udfordringer, bruges de mest til lokal overvågning. 
\par
Skulle det ske at der gik noget galt i en dæmning, og at det ikke blev opdaget i tide, kan det ske at mennesker bliver fanget på grund af en oversvømmelse. Her kan dronerne bruges til hurtigt at få informationer om, hvordan uheldet eskalerer, og man kan derfor få evakueret faretruede områder hurtigere, end hvis man med menneskekraft skulle ud og få overblik over situationen. Selvom en oversvømmelse ofte er en langsomt eskalerende situation, er der stadig risiko for at ikke alle folk bliver evakueret, og her kan droner bruges til at søge efter mennesker.
% Droner bruges altså i denne sammenhæng da....

\subsection{Skovbrande}
Det sidste eksempel der vil blive gennemgået, er skovbrande. Her bruges dronerne til at finde såkaldte \textit{hotspots}. Et hotspot er et område der er varmere end de omkringliggende områder, som for eksempel hvis der er opstået en brænd i midten af en skov. Dronerne flyver over skoven på en prædefineret rute, og leder efter hotspots. Denne metode med dronerne har både fordele og ulemper, blandt andet er de ret dyre at bruge i forhold til bemandende fly [KILDE], men til gengæld er de nemmere og hurtigere at bruge. Hvis dronen finder et hotspot, rapporteres det direkte tilbage til kontrolstationen, og anses dette for at være en reel fare, kan brandvæsnet kontaktes med det samme. 
% Droner bruges altså i denne sammenhæng da....

% Inden for andre områder erstatter man altså mennesker med droner, kunne denne taktik bruges inden for oversvømmelser? Altså, tage mennesker ud af eksisterende løsninger. 

