% Noget med droner

\section{Permanente løsninger}\label{sc:permanente}
I tilfælde af en stormflod kan der ske oversvømmelser både ved kysten og længere inde i landet hvor vandet presses op gennem vandløb \todo{}[KILDE]. Der er således potentielt store områder med bebyggelse, som er i fare for beskadigelse. En tilgang til dette problem er at bygge permanente strukturer i landskabet, der skal forhindre oversvømmelserne, eller at konstruere bygninger med henblik på, at de skal kunne modstå vandet. I dette afsnit skitseres nogle af disse løsninger. 

\subsection{Løsninger ved kysten og vandløb}
% Havdiger
% http://kysterne.kyst.dk/hvad-er-diger.html
I tilfælde af stormflod, hvor lavtliggende kystområder risikerer at blive oversvømmet, er det vigtigt at overveje, hvordan man beskytter bebyggelse i disse områder. Dette kan gøres på flere måder. 
Mange steder i Danmark er der bygget diger til at forhindre oversvømmelser i arealerne bag digerne. Disse er lange forhøjninger i terrænet, der fungerer som en barriere for havvand, der ellers ville oversvømme det bagvedliggende land. De bygges ofte lidt væk fra kystlinjen og med skrånende sider for at minimere erosion fra vand og bølger \cite{havdiger}. 
\par
% Højvandsmure
% https://www.klimatilpasning.dk/viden-om/teknologi/stigende-havspejl/hoejvandsmurspunsvaeg/
Især på havne i byer er højvandsmure også en mulighed. Her bygges der lave mure, ofte af beton, der skal holde vandet ud af byen bagved, hvis vandstanden stiger. En sådan mur kan også bruges til rekreative formål, såsom bænke, og der kan være åbninger med porte så man kan komme gennem muren \cite{hoejvandsmur}. 
\par
% Stormflodsbarrierer 
% http://kysterne.kyst.dk/stormflodsbarrierer.html
Diger og højvandsmure er eksempler på konstruktioner på land, der skal forhindre oversvømmelse af bagvedliggende arealer. Andre konstruktioner bygges også ud i vandet, eksempelvis på tværs af fjorde eller vandløb. Et eksempel på sådanne konstruktioner, er stormflodsbarrierer. Disse kan bygges på flere forskellige måder, men består ofte af en form for port der kan lukkes af for at undgå oversvømmelser, når vandstanden stiger \cite{stormflodbarriere}. 
\par
% Sluser og pumpestationer
% https://www.klimatilpasning.dk/viden-om/teknologi/stigende-havspejl/sluser-og-pumpestationer/
Til forskel fra stormflodsbarrierer, der er delvist mobile, er slusedæmninger faste installationer, der bygges på tværs af vandet, eksempelvis ved et havneindløb eller vandløb. Disse kan også være åbne eller lukkede for enten at lade vande løbe igennem eller holde det tilbage. Der kan også være en pumpe tilknyttet for at kunne pumpe vand tilbage ud i havet ved for høj vandstand \cite{sluse}. 
% diverse dæmninger (fx weirs)
% kanaler
% https://www.tv2fyn.dk/artikel/ny-kanal-skal-redde-bogense-fra-oversvoemmelser
% KYST:
%   beach nourishment (kystfodring)
%   barrier islands
%   bølgebrydere?
% VANDLØB:
%   flodplains
%   tidevandsport /tide gate

\subsection{Sikring af bygninger}
% Forhøjet fundament. Placering af huse, højere terræn (naturlig eller kunstig)
% https://www.klimatilpasning.dk/sektorer/kyst/risiko-for-oversvoemmelse/planlaegning/
Den simpleste måde at undgå skader ved oversvømmelser er ved ikke at bygge huse i de områder, hvor der er øget risiko for oversvømmelse. Hvis der alligevel skal bygges i sådanne områder kan man lægge husene på højere terræn, enten naturligt eller kunstigt opbygget, eller man kan bygge dem med et forhøjet fundament \cite{planlaegning}. 
\\\\
% vandtæt dug
% https://www.klimatilpasning.dk/viden-om/teknologi/stigende-havspejl/vandtaet-dug/
En anden strategi er at søge for, at huset kan modstå en eventuel oversvømmelse. Dette kan gøres på forskellige måder. Der kan eksempelvis graves en boks ned rundt om huse, der indeholder en sammenrullet vandtæt dug. Denne dug vil så løftes op med vandet og rulles ud op ad husets ydervæg, så den forhindrer vandet i at komme ind i huset \cite{vandtaetdug}. 
\par 
% Vandsikring i selve huset. 
Der findes også andre tilføjelser til selve huset kan også hjælpe med at beskytte mod vandskader. Dette kunne eksempelvis være vandtætte døre og vinduer \cite{barrierekatalog}. Ved oversvømmelse kan vand fra kloakken også trænge sig ind i huset gennem afløb, selv hvis huset er bygget på et relativt højt terræn \cite{stormflodssikring}. For at forhindre dette kan der eksempelvis installeres kontraventiler \cite{planlaegning} eller højvandslukke \cite{forebyg} i afløbet. 
% porte
% ydermure der kan tåle vandet

\subsection{Delkonklusion}
Fælles for de permanente løsninger til at beskytte mod oversvømmelse og medfølgende vandskader, er at de oftest implementeres af professionelle. Løsningerne er desuden ofte specialiserede til den enkelte situation de forekommer i og implementeringen kræver således en vis fleksibilitet og fantasi. Mens automatisering, om det er i form af droner eller ej, muligvis vil kunne øge effektiviteten af opsætningen af disse løsninger, er der ikke fundet nogen konkrete og generelle problemer.
% hvordan bruges droner i konstruktion (er der et problem her et sted?)





% Fravalg af permanente: Bygges kun én gang. Speciel løsning til hvert sted.


% Links:
% https://www.tvmidtvest.dk/artikel/holstebro-klar-med-milliondyr-sikring-mod-oversvoemmelser
% http://stormflodssikring.dk/wp-content/uploads/2019/01/Barrierekatalog-Stormflodssikring-ApS.pdf

