\section{Hvor skal digerne ligge}\label{sc:digebeliggenhed}
Når der skal bygges diger af sandsække for at forhindre eller mindske oversvømmelser i forbindelse med en stormflod, er det vigtigt at vide hvor de skal lægges for at være mest effektive. Dette er et problem der kan løses enten af den enkelte drone i sværmen eller af en central computer, der kan videregive de nødvendige informationer til dronerne.  \\\\
Geodatastyrelsen har udarbejdet Danmarks Højdemodel, der er et kort, som indeholder detaljerede data om højdeforskellene i Danmark og som er frit tilgængeligt for alle \cite{dkhoejdemodel}. \par Modellen bruges af flere digitale værktøjer, til at modellere eksempelvis hydrologi. SCALGO Live er et sådant værktøj \cite{scalgo}\cite{omscalgo}. Det findes online og giver brugeren mulighed for at modellere hvordan vand strømmer over terrænet. Det er således muligt at identificere områder med øget risiko for oversvømmelse, samt hvor det er relevant eksempelvis at sætte ind med midlertidige sikringer i tilfælde af en stormflod\cite{scalgo}\cite{omscalgo}. \par
SCALGO Live er tidligere blevet brugt til dette formål ved en stormflod i 2017, hvor det blandt andet blev brugt til undersøgelse af effekten af midlertidige sikringer, samt til at informere borgerene i forbindelse med placering af sandsække \cite{alarmscalgo}.\\\\
Der eksisterer altså allerede værktøjer til at finde ud af hvor digerne skal ligge. Dette undersøges i dag manuelt, men med videreudvikling af værktøjer som SCALGO Live vurderes det at det vil være muligt også at automatisere og dermed effektivisere denne undersøgelse. Af denne grund antages det i dette projekt at beslutninger om placeringen af digerne forgår udenfor dronerne. Information om denne placering vil være tilgængelig for den enkelte drone i simulationen, som så kan bruge denne viden til opbygningen af digerne.  

%SCALGO opland