\chapter{Problemformulering}\label{ch:problemformulering}
% Kort opsummering og hvorfor programmet skal laves
Som nævnt i indledningen, bliver teknologi ofte inspireret af naturens sværme. Det er derfor vigtigt at indsamle data og viden til forskning. Dette kan lede til praktiske problemer, da dataindsamling er ressourcekrævende. Her er computersimulering blevet et værktøj som i visse tilfælde formindsker forbruget af ressourcer, såsom tid og penge.
\\
Simulationer bliver brugt til at generere modeller af sværmes adfærd på baggrund af indsamlet data. Til dette findes der bl.a. regelbaserede simulationener, som eksempelvis bruger Boid-modellen. Problemet med en regelbaseret tilgang er, at programmøren skal definere adfærden, før simulationen kan finde sted. Det er derfor svært at bruge en regelbaseret tilgang hvis formålet det er, at finde frem til en adfærdsmodel uden at reglerne er kendte. Alternativet til en regelbaseret tilgang er en læringsbaseret fremgangsmåde som selv finder frem til reglerne, der definerer sværmedynamikken. Der er dog det problem, at de læringsbaserede metoder, som bliver brugt i forskningen, ofte har et black box problem. Outputs i form af vægtede sandsynligheder mv., giver kun begrænset information om, hvorfor det enkelte individ handler som det gør.
\\
Imidlertid eksisterer mindre brugte læringsbaserede metoder, som der ses et potentiale i. På baggrund af dette bliver formålet for dette projekt at udvikle en læringsbaseret sværmesimulation, som overkommer begrænsningerne ved den regelbaserede tilgang såvel som black box problemet. Det er blevet vurderet, at genetisk programmering udviser det største potentiale i denne sammenhæng, og løsningen vil derfor blive implementeret med denne metode.
\par 
På baggrund af problemanalysens afgrænsninger, kan følgende problemformulering fremsættes:
\par
\textit{Kan der udvikles et værktøj, der ved hjælp af genetisk programmering kan simulere sort sol?}
\par
Værktøjet tiltænkes at være et supplement til nuværende løsninger, men ikke en erstatning, da det ikke kan garanteres at være lige så præcist, som nuværende løsninger.
\\
Grundet et kortere projektforløb og begrænsede ressourcer bliver der lavet et proof-of-concept som afgrænses til en udvalgt art. I dette tilfælde vælges stære, grundet en større tilgængelighed af viden til at vurdere kvaliteten af simulationen.
\par

% Nuværende løsninger (Supplement)
% Der findes dog allerede løsninger, som finder frem til reglerne hos en given sværm. Dog findes der typisk fordele og ulemper ved forskellige simulationsmetoder, som der eksempelvis er ved regelbaserede og læringsbaserede tilgange. Det vil derfor ikke skade at udvide forskernes muligheder med en ekstra løsning som beskriver sværmes adfærd. Derfor vil dette program blive et supplement som eksempelvis kunne bruges til at sammenligne resultater. 




% Hvordan kan man ved hjælp af en læringsbaseret tilgang, simulere sværme?

% Det er problem at indhente data fra virkeligheden, altså feltarbejde, og vi vil undersøge om en læringsbaseret tilgang kan hjælpe på dette.

% Kan man ved hjælp af genetisk programmering, udvikle et program, der imiterer stæreflokke i virkeligheden? (Kan man gøre det bedre?)

% Hvordan kan man udvikle et værktøj, til at opbygge modeller af stære flokke?




