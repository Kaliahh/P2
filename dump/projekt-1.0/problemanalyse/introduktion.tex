\chapter{Problemanalyse}\label{ch:problemanalyse}
I daglig tale og indenfor biologien er en sværm ofte en betegnelse for en gruppe af mindre dyr, eksempelvis insekter eller fugle, der bevæger sig i samlet flok \cite{definition}. Når sværme omtales i en bredere kontekst, eksempelvis i forbindelse med modellering, er definitionen dog bredere. Her kan udtrykket også inkludere grupper af større dyr, mennesker, robotter, partikler eller andre ikke-biologiske agenter \cite{rosenberg2012}. Sværme defineres her ud fra bevægelsesmønstrene, frem for de dyr, sværmen består af \cite{biofoundations}. Hovedprincippet i denne type af sværmdannelse er, at der ikke træffes nogen centrale beslutninger for sværmens bevægelse som helhed, men at den opstår af decentrale interaktioner mellem dens medlemmer. Sværme defineres med andre ord ud fra, at deres dynamik er selvorganiserende \cite{kelso1995}. I denne rapport bruges udtrykket sværm efter den sidste af disse definitioner. 