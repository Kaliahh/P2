\section{Hvordan simulerer man sværme?} \label{ch:hvordansvaerme}
Meget forskning inden for sværmeadfærd involverer indsamling af store mængder data, der derefter oversættes til en model og simulation af den undersøgte sværm. Dette kan foregå på flere måder. Dataindsamlingen involverer i dag ofte optagelser eller billeder af en sværm, der kan analyseres og oversættes til en model med mere eller mindre hjælp fra en computer [CITATION]. Denne hjælp kan være i form af en  regelbaseret tilgang eller en læringsbaseret tilgang. Enten oversætter forskerne selv deres data til en model eller de udnytter en form for maskinlæring til at oversætte deres data for dem. Et eksempel på forskning der bruger en regelbaseret tilgang er en undersøgelse af hvordan en art af myrer (ARTSNAVN) finder et nyt bo. her har forskerne markeret de enkelte myrer og filmet deres adfærd når de søger efter det bedste sted at flytte kolonien hen. herefter har de opstillet en model som beskriver de handlinger og valg den enkelte myrer tager [CITATION].
Et eksempel på den læringsbaserede tilgang er en undersøgelse hvor... noget med en probabilistisk model blabla..... [CITATION.]

% Stereo method, Shadow method https://www.sciencedirect.com/science/article/pii/000334726590117X

% Den gode quote: https://www.technologyreview.com/s/415022/first-simulation-of-the-flocking-behavior-of-starlings/

% Artikel hvor de bruger data fra den forrige artikel: https://arxiv.org/ftp/arxiv/papers/0908/0908.2677.pdf

% https://people.mpi-inf.mpg.de/~mehlhorn/SeminarEvolvability/CuckerSmale.pdf
% http://gamma.cs.unc.edu/VisInsectSwarms/VisInsectSwarms-paper.pdf
% https://www.sciencedirect.com/science/article/pii/S0003347205002332#fig1
% https://ac.els-cdn.com/S0957417413003394/1-s2.0-S0957417413003394-main.pdf?_tid=5800edaa-843b-4401-999b-5fa6cb423827&acdnat=1551788711_b224d586a988bdae69fbd3fe9f10ca52