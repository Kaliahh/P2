\section{Sværme}
I dette afsnit foretages en systematisk kortlægning af de typer af sværmeadfærd, som finder sted i naturen. På baggrund af dette, udvælges en relevant case som fokus for resten af projektet. 
% I dette afsnit skal der udvælges en sværm, baseret på 
% Overordnet definition og vores definition for resten af rapporten
% Skal der noget med sværmintelligens med her?

% En sværm er ifølge \cite{definition} en \textit{"større gruppe af mindre dyr, oftest insekter eller fugle, der tæt samlet bevæger sig af sted"}, når der i nærværende projekt tales om sværme, er det ud fra denne definition. 

\subsection{Typer af sværme}
% Eksempler på typer af sværme og organismer indenfor hver type
Der er mange forskellige organismer der danner sværme og formålet med denne sværmdannelse, kan variere fra art til art. Fælles for dem er, at der ikke træffes nogen centrale beslutninger for sværmen, men at sværmens reaktioner og bevægelsesmønstre skyldes beslutninger som de enkelte individer træffer ud fra de lokale informationer de har adgang til. Sådanne informationer kan komme fra andre individer i sværmen eller fra omgivelserne. \\
Årsagen til sværmdannelsen og formålet med sværmen spiller dog en rolle i hvilken type sværm der dannes. I dette afsnit gennemgås nogle af disse formål og med eksempler på arter og mekanismer til denne sværmdannelse.

\subsubsection{Fouragering}
Én type sværm er, når en større gruppe individer af samme art lever sammen i en koloni. Denne type sværm findes oftest blandt insekter, hvor bl.a. myrer og bier er velkendte eksempler. Når så mange individer lever sammen, er det nødvendigt at finde og bringe føde med hjem til kolonien. Dette sker mest effektivt hvis der er kommunikation mellem individerne i kolonien. Dette kaldes også rekruttering og kan foregå på forskellige måder, indirekte eller direkte. Honningbier bruger eksempelvis direkte rekruttering. Hvis en bi finder en god fødekilde kan den videregive informationer om retning og afstand til kilden, samt dens kvalitet, til andre bier i kolonien ved at lave en bestemt dans. Der er altså tale om en direkte kommunikation mellem individer \cite{beekman2008biological}. 
\par 
Indirekte rekruttering sker, uden at individer kommer i direkte kontakt med hinanden. Et eksempel på dette kunne være flere arter af myrer, der kommunikerer lokationen af en fødekilde til andre fra kolonien ved at efterlade et duftspor, som de kan følge. Denne metode fører også ofte til, at sværmen finder den korteste vej til fødekilden eller går til den kilde, der er tættest på. Dette skyldes et positivt feedback loop, hvor sporet til fødekilden forstærkes når flere myrer følger det \cite{beekman2008biological}. 
\par
Fælles for begge metoder for rekruttering er, at det foregår individ til individ, uden at hele kolonien eller sværmen bliver involveret direkte. Der sker eksempelvis ikke en egentlig evaluering af hvilken fødekilde der er tættest på myretuen, men den bliver stadig valgt ved, at information gives videre fra individ til individ \cite{beekman2008biological}. 
% Måske afgrænse her?

\subsubsection{Flytte bo}
Når en sværm i form af en koloni af insekter skal flytte bo, enten fordi deres gamle bo er blevet ødelagt eller fordi det er for småt til kolonien, skal kolonien samlet beslutte, hvor de tager hen. Dette foregår bl.a. hos honningbier, der udfører eftersøgningen og valget af et nyt bo på en tilsvarende måde til, hvordan de fouragerer. Der ses dog andre metoder blandt myrer, hvor arten \textit{Themnothorax albipennis} ikke benytter sig af et duftspor for at lede kolonien hen til et nyt potentielt bo, til forskel fra hvordan myrer ofte finder føde. Disse myrer danner relativt små kolonier, hvor de der tager ud og leder efter et nyt bo og vender tilbage til kolonien, hvis de fandt et sted de vurderede til at være godt. Herefter leder de en anden myre hen til det potentielle nye bo for selv at kunne vurdere det. Jo bedre stedet er, jo hurtigere tager de hjem og viser en anden myre vejen derhen. Der vil således samles flere og flere myrer ved det potentielle nye bo, hvis det er det bedste, der er fundet. Når en stor nok del af kolonien er samlet er beslutningen truffet og de myrer, inklusiv dronningen, der stadig er tilbage ved det gamle bo, bliver hentet og båret til det nye bo. Denne myreart er velstuderet og deres adfærd er blevet modelleret og sammenlignet med tilsvarende situationer med virkelige myrer \cite{beekman2008biological} \cite{pratt2005agent}. 

\subsubsection{Bygge bo}
Flere arter, der danner sværme bygger også komplekse bo, især blandt de sociale insekter. Her er et eksempel termitterne, der bygger store bo. Forskellige faktorer, som eksempelvis lys, varme og luftstrømme, kan påvirke den enkelte termit til at bygge og forbedre boet. Ligeledes kan uregelmæssigheder i boets struktur få visse arter af termitter til at bygge. Det er også normalt at termitter der bygger efterlader et spor at feromoner, der får flere termitter til at hjælpe til eller fortsætte arbejdet \cite{farago2014biocommunication}.  Der er således paralleller til hvordan eksempelvis myrer fouragere. \par
Mens de fleste arter der danner kolonier og bygger et bo er insekter, finder der også en art af fugl, republikaneren (\textit{Philetairus socius}), der bor i kolonier og bygger enorme reder i fællesskab. I disse kolonier vil der være nogle individer der bygger mere på reden end andre. De fugle der bygger meget er også mere aggressive overfor andre fugle i kolonien og hvis en fugl der ikke bidrager med så meget bliver offer for denne aggression er de mere tilbøjelige til at hjælpe med at bygge på reden. Dette er således en mekanisme hvorigennem den fælles rede bygges og gøres større \cite{leighton2016sociable}.

\subsubsection{Undgåelse af rovdyr}
For nogle dyrearter kan der opstå sværmedynamik, når grupper af artsfæller organiserer sig for at undgå en fælles ydre fare i form af rovdyr. Koordineringen dyrene imellem giver her nye overlevelsesmuligheder for det enkelte dyr \cite{chen2014}. Eksempelvis samler mindre fisk og fugle sig instinktivt i større stimer og flokke - bl.a. sardiner og stære \cite{fishschooling}\cite{sortsol}. De enkelte dyr følger tilsyneladende simple regler: De fjerner sig navnlig fra rovdyr og undgår kollision med naboer, samtidigt med at de holder sig tæt på, og følger deres artsfæller \cite{reynolds1987}.
\par
På trods af de simple regler, observerer flere forskere, at der her kan opstå komplekse kollektive adfærdsmønstre, som giver det enkelte dyr nye overlevelsesmuligheder. Ifølge \textit{many eyes} hypotesen, øger kombinationen af mange årvågne øjne og koordinerede bevægelser, sværmens evne til at spotte og undgå rovdyr \cite{fishschooling}. De kan potentielt undgå mange angreb fra flere sider ad gangen, og med en betydeligt lavere responstid, end hvad tilsvarende enlige dyrearter præsterer \cite{hoaremfl2004}.
\par % Det her stykke skal lige omstruktureres 
Ifølge \textit{predator confusion effect} hypotesen giver dyrenes massebevægelser mulighed for at forvirre et rovdyr. Det kan være svært at fokusere på et enkelt bytte i virvarret \cite{jeschke2007}. Andre har argumenteret for, at dyreflokke på landjorden, i form af gnuer, zebraer mv., bruger en lignende form for kollektiv intelligens til at få øje og reagere adaptivt på rovdyr i tide. Ved disse arter spiller alarmkald også en betydelig rolle \cite{rosenberg2012}. Imidlertid er der flere hypoteser i spil på området og ingen gennemgående konsensus - Hverken i forhold til forklaringsmodellerne eller deres generaliserbarhed på tværs af flere arter \cite{conradt2011}. 

\subsubsection{Angreb og forsvar}
Andre former for sværmeadfærd i naturen forbedrer dyrearters kollektive evne til at angribe eller forsvare sig mod andre dyr. Her kan der opstå muligheder for at overrumple ellers uoverkommelige fjender.
\par
Flere sociale insekter udviser denne adfærd. Bier kan eksempelvist organisere sig i guerrilla-lignende formationer, som giver maksimal dækning af et luftområde med en samtidig høj sværmdensitet, så fjenden effektivt angribes fra flere sider på samme tid \cite{arquilla2000}. En bestemt østasiatisk honningbi er i stand til at overrumple gedehamse ved komplet at omslutte og koge den i en såkaldt \textit{ball of death} \cite{ugajin2012}. Myrer, som ellers ofte bevæger sig i rækker over længere afstande, kan hurtigt foretage et markant formationskifte, hvis de går til angreb på andre insekter ved såkaldte \textit{swarm raids} \cite{holldobler1994}. Myrene er her i stand til at organisere angreb i bølger, så dyrene i front løbende får forstærkninger bagfra, og målet da bliver løbet over ende \cite{edwards2000}. % Pulsing i militærtermer
\par
\textit{Mobbing} er en anden form for kollektiv adfærd, der typisk bruges til at beskytte afkom eller stjæle føde fra andre dyr \cite{alcock1998}. Den spiller sig oftest ud hos fugle såsom måger og krager, som udsætter målet for angreb fra flere sider i form af høje skræppelyde, ekskrementer, opkast og flyvende taklinger (\textit{dive bombs}) \cite{altmann1956}. 
\par
Umiddelbart kan adfærden for større fugle synes mere opportunistisk og ukoordineret end egentlig sværmeadfærd \cite{arquilla2000}. Imidlertid er der ved mindre fugle, eksempelvis svaler, observeret mønstre i adfærd, der typer på en sværmedynamik, som forøger angrebets effekt. Fuglene flyver rundt om målet i ellipseform og følger et periodisk angrebsmønster, så målet er under konstant pres fra forskellige sider og derfor har svært ved at gengælde angrebet \cite{bardi2010}. 

% Afgrænsning?

\subsubsection{Når sværmen flytter sig}
Der er overordnet set to måder en sværm kan bevæge sig fra et sted til et andet. Enten medvirker alle individerne i gruppen til beslutningen om, hvilken retning sværmen bevæger sig i, eller der er en mindre gruppe af individer, der ved hvor de skal hen, og leder resten derover. Hvilken metode en sværm bruger, vil afhænge af hvilken art der er tale om \cite{beekman2008biological}.
\par
Hvordan en sværm af honningbier bevæger sig er forklaret af en hypotese om, at nogle enkelte individer ved, hvor de skal hen. Disse individer viser de andre vej, ved at flyve hurtigt gennem sværmen i retning af deres mål. Dette er dog kun en hypotese, selvom der er studier der tyder på, at det er rigtigt \cite{beekman2006}. 
\par
En art af fårekyllinger, \textit{Anabrus simplex}, danner sværme i bånd, når de migrerer. Det vides stadig ikke, hvad mekanismerne bag denne sværmdannelse præcist er, men undersøgelser tyder på, at bevægelserne af sværmen ikke følger en eller nogle få individer. Bevægelserne er derimod et resultat af, hvordan alle individerne bevæger sig og reagerer på dem der er i nærheden. Andre undersøgelser har vist, at overlevelsen indenfor sværmen kan være væsentligt større end udenfor sværmen, så beskyttelse mod rovdyr kan være en årsag til at fårekyllingerne samles \cite{beekman2008biological}. Samtidig drives sværmen dog fremad af kannibalisme. Da fårekyllingerne i sværmen ofte lider af mangel på næring som protein og salt, er den nemmeste kilde til disse ressourcer ofte de andre individer i sværmen. Der er således farligt for det enkelte individ at stå stille i sværmen, da de risikerer at fårekyllinger, der kommer bagfra, spiser dem. Dette er med til at drive sværmen fremad \cite{beekman2008biological}. 
\par
Som nævnt i indledningen (kapitel \ref{ch:introduction}) bygger hærmyrer broer over kløfter i deres rute. Dette fungerer ved at en myre der når kløften standser op, mens myrerne bagved fortsætter, så de ender med at gå hen over den første myre. Denne myre står stille når der er nogen der går på den og er således det første led i broen. den næste myrer der heller ikke kan nå over kløften gør således det samme indtil broen strækker hele vejen over kløften \cite{armyants}.


\subsubsection{Valg af en type}
% Konklusion på afsnittet. Hvilken type vælger vi? Og hvorfor (Se tekniske krav ovenfor + "Brainstorm over typer" på drevet)? 
Efter denne kortlægning af naturens sværmetyper, er det muligt at vælge én af dem til dette projekt på baggrund af nogle rent tekniske overvejelser. Vores interne eksperimenter med prototyper og valget af en læringsbaseret metode medfører her tre [eller gerne flere] krav til den valgte type, som begrunder afgrænsningen. 
\par
% Vi vil helst minimere behovet for at simulere fysiske interaktioner
I vores interne prototypeudvikling fandt vi for det første et problem med cases, hvor der skal modelleres mange fysiske interaktioner mellem dyrene og omverdenen. Vi vurderede her, at de tekniske udfordringer med at skabe realistiske fysiske interaktioner i et virtuelt miljø, ville fylde for meget i forhold til dette projekts hovedformål. Herfra kom kravet om, at disse fysiske interaktioner ikke skal være væsentlige at modellere for den valgte type. Dette begrunder fravalget af to typer: Angreb og bobygning, da omdrejningspunktet for de to typer henholdsvis er fysiske interaktioner med andre dyr og med byggematerialer. 
\par
% Helst så korte læringsrunder som muligt og så små udfordringer som muligt med overfitting
To andre krav lægger sig til valget af en læringsbaseret tilgang. Hvis der skal vælges mellem de resterende typer, er det for det første ønskeligt at have så hurtige og simple læringsrunder som muligt. Dette muliggør alt andet lige løsningen af mere komplekse optimeringsproblemer inden for kortere tid. Det andet krav lægger sig til risikoen for \textit{overfitting} ved brug af en læringsbaseret metode. Overfitting er, når et læringsbaseret systems evne til generel problemløsning begrænses af, at læringen har fundet sted på et ikke-repræsentativt datasæt \cite{hawkins2004}. Hvis nogle simulerede myrer eksempelvis trænes til at navigere på den samme rute i en bestemt skov, kunne dette måske indskrænke deres evne til at navigere alle andre steder.
Hvis der praktisk talt er et uendeligt antal scenarier, som skal dækkes ind for at modvirke overfitting, bør den pågældende type sværmeadfærd om muligt fravælges. 
\par
Disse to krav begrunder fravalget af tre typer: Fouragering, boflytning og migration. Sværmeadfærden kan nemlig her udspille sig over større geografiske afstande, hvor simulationen af disse giver flere beregninger og derfor længere læringsrunder. Der er ved disse typer også praktisk talt uendelig variation, når det kommer til mulige konfigurationer af miljøet, som denne form for adfærd kan udspille sig indenfor. 
\par
Dette efterlader en resterende type i form af undgåelse. Adfærden imødekommer ovenstående tekniske krav til forskel fra de andre typer. Direkte fysiske interaktioner er ikke styrende for denne type adfærd. Derudover kan adfærden simuleres inden for et repræsentativt miljø og med læringsrunder, der ikke kompliceres af geografiske afstande eller andet. Altså vil projektet implementere sin simulation med en konkret case inden for denne type. 



\subsection{Dyrearter inden for den valgte type}
Dette projekt går ud på at undersøge en alternativ metode til et tidligere behandlet problem. For at kunne vurdere kvaliteten af vores løsning er det derfor vigtigt at kunne sammenligne den med andre lignende implementeringer.
Jo mere konkret projektets valgte naturfænomen til simulering er, desto mere konkret bliver sammenligningsgrundlaget. Derfor er det en fordel at vælge en bestemt dyreart som case for projektet frem for at arbejde videre med en abstrakt type.
\par
Flere dyrearter udviser sværmeadfærd for at undgå rovdyr. Den ideelle art at udvælge til dette projekt er velundersøgt både inden for natur- og simulationsforskningen. Dette muliggør evalueringer af vores løsnings kvalitet over for foreliggende viden om naturfænomenet såvel som tilsvarende tekniske løsninger. 
\par
Størstedelen af forskningen om emnet drejer sig som nævnt enten om fugle- eller fiskearter og kun i begrænset grad om landlevende dyr. Hver af kategorierne \textit{"fish*"} og \textit{"bird*"} får flere end dobbelt så mange søgeresultater som \textit{"herd*"} inde på AAU's litteraturdatabase, når der søges efter \textit{"swarm*"} og/eller \textit{"simulat*"} og/eller \textit{"predator*"} \cite{AUB}. Da vi inden for dette udbud observerer en begrænset mængde litteratur om specifikke landlevende dyrearter, vil valget mellem konkrete dyrearter i stedet foregå inden for kategorierne fugl og fisk.


\subsubsection{Valg af en dyreart}
Valget mellem fugl og fisk kan også foretages på baggrund af bibliometri. Med ovenstående søgeord på samme database fandt vi, at antallet af emnerelevante artikler om fisk var ca. 15\% større end antallet af artikler om fugle \cite{AUB}. Imidlertid fandt vi ved gentagne søgninger et problematisk mønster i udbuddet af litteratur om fisk i AAU's database: Antallet af forskellige undersøgte arter var tilsyneladende meget højt, men dette medførte, at de enkelte arter ikke var lige så undersøgte som i fuglenes tilfælde, hvor udbuddet af undersøgte arter tilsyneladende var mindre. Denne større bredde i litteraturen om fisk kunne måske delvist hænge sammen med, at biodiversiteten i havet er langt højere end i luften \cite{jakhsha2010}.
\par
Da målet frem for alt var at udvælge den mest velundersøgte art, faldt valget altså i stedet for på en art inden for fugleverdenen. Inden for simulationer af fugle, var den mest undersøgte art ifølge AAU's database \textit{Sturnus vulgaris}, den europæiske stær, som gav flest søgeresultater. Dette kunne måske hænge sammen med, at arten er i top 10 over de mest hyppigt forekommende i verdenen \cite{birdtop10}. Blodnæbsvæveren, rødvingetrupialen og den rustkronede træspurv udviser også undgående sværmeadfærd og er mere hyppigt forekommende end stæren på samme rangliste. Imidlertid er der mere end ti gange så mange artikler om simulationer af stære end disse andre arter tilsammen i databasen \cite{AUB}. Altså blev stæren dette projekts valgte case, fordi udbuddet af litteratur om arten tænktes at give den bedste mulighed for at vurdere projektets implementering. 



\todo[inline]{}