\chapter{Indledning}\label{ch:introduction}

% Vi sporer os ind på sværme-fænomenet med nogle konkrete eksempler
I naturen optræder flere dyrearter i sværme. Fiskestimer undviger hajer med velkoreograferede svømmemanøvrer \cite{fishschooling}. Enorme flokke af stære danner sort sol \cite{sortsol}. Hærmyrer bygger broer af hinandens kroppe \cite{armyants}. Sværmeadfærd tjener et formål for dyrs overlevelse, alt fra undgåelse af rovdyr til nye bevægelsesmuligheder \cite{biofoundations}.
\par
% Sværme er relevante at vide noget om. 
% Mennesker er begyndt at interessere sig for denne adaptive adfærd. 
Indsigt i sværmedynamik tænkes bl.a. at være relevant i beskyttelsen af økosystemet. Flere sværmende insektarter med vigtige roller i bestøvningen af blomster m.m. er på vej til at uddø \cite{ripinsects}. Implementeringen af bæredygtige løsninger - det være sig robotinsekter, fredning af bestemte naturområder el. andet - kan understøttes af viden om sværmedynamik \cite{beespatent}\cite{swarmdecline}.
\par
Andre forsøger at oversætte fra natur til teknologi. Man har bl.a. ladet sig inspirere af dyrs sværmeadfærd i designet af selvorganiserende pixels til højdefinitionskameraer, og flyvende drone-formationer til militæret \cite{camera}\cite{pentagon}. Der er altså flere mulige gevinster for mennesker ved at forstå dyrs sværmeadfærd.
\\\\
% Sværme er svære at forstå i detaljer
Viden om de lokale interaktioner mellem dyrene i en sværm er vigtig for at forstå sværmedynamikken til fulde, da sværmen som helhed styres af disse, og ikke af en central dirigent \cite{selforganimals}.  
Imidlertid kan det være svært at beskrive sværme på dette niveau. Det kan for det første skyldes praktiske problemer med at indsamle den nødvendige data i felten såvel som i laboratoriet. Insektsværme i luften kan eksempelvis være svære at observere i detaljer \cite{swarmsaredifficult}.\\ 
Et andet problem lægger sig til den \textit{ikke-lineære} dynamik, som typisk er grundlaget for sværmeadfærd. Her påvirker de lokale interaktioner hinandens effekt på systemet som helhed \cite{swarmsarenonlinear}. Kortlægningen bliver da langt mere vanskelig end ved et lineært system \cite{nonlinear}, da det er mere komplekst.
\par
% Computersimulationer som muligt alternativ til felt- og laboratoriestudier
Denne type udfordringer gør det relevant at undersøge alternative metoder til felt- og laboratoriestudier. Computersimulering af sværme er en relativt ny tilgang med flere fordele. Eksempelvis er forbruget af tid og penge pr. forsøgsrunde relativt lavt, så snart systemet er sat op  \cite{simulationintro1}\cite{simulationintro2}. Der åbnes altså op for et større antal forsøgsrunder med nye muligheder for at bestemme testmiljøet under afprøvning. Dette gør det nemmere at kortlægge ikke-lineære interaktioner \cite{nonlinear}. Derudover kan visse etiske regler for eksperimenter med levende dyr tilsidesættes, og derfor opstår der helt nye muligheder \cite{animalethics}. Computersimulering kan altså være værd at undersøge nærmere af flere grunde.
\\\\
% Den initierende undren
Vores initierende undren for dette projekt er hermed: 
\par
\textit{Hvordan kan man ved hjælp af computersimulering finde indblik i grundlaget for sværmeadfærd i naturen?}
\par
Det vil her være relevant at kigge på fordelene og ulemperne ved en regelbaseret tilgang, som ofte finder anvendelse. Denne metode kræver en forståelse for de fænomener man vil simulere, og som tidligere beskrevet kan denne viden være svær at opnå. Det vil derfor være interessant at undersøge alternative metoder inden for kunstig intelligens og maskinlæring til at modellere og simulere sværmeadfærd. 
\par
Dette tekniske problem er relevant at belyse pga. de ovennævnte mulige gevinster ved en bedre forståelse af sværmedynamik. Disse involverer både nye muligheder for at bidrage til beskyttelse af økosystemet og for ny teknologi til almen nytte, som nævnt tidligere.
\par
% Meta-kommunikation
Problemanalysen er struktureret således, at efter en almen definition af sværmeadfærd, undersøges eksisterende løsninger inden for computersimulering af fænomenet. Herefter identificeres en begrænsning ved nuværende simulationsteknologi, når denne anvendes til at kortlægge naturfænomer. Muligheden for at supplere disse løsninger med mindre brugte metoder fra en alternativ tilgang fremsættes, hvorefter en bestemt dyreart udvælges til at afprøve mulighederne inden for denne. På baggrund af problemanalysen opstilles herefter en problemformulering, som danner rammer for projektets løsningsdel.




% Sværm, oversættelse fra natur til computersimulation
% Initierende undren:
    % Hvordan kan man ved hjælp af maskinlæring finde indblik i, hvilke parametre og faktorer der får en bestemt sværm til at opstå? Et delproblem hertil er, at finde hvilke parametre og faktorer i naturen, som har indflydelse på dette.

% Oversættelse fra natur til maskine \\
% Hvad kan det bruges til? (Inspiration fra naturen) \\
% Valg af simulering, som undersøgelsesmetode.\\